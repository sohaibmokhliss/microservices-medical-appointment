\chapter{Résultats et Discussion}

\section{Introduction}

Ce chapitre présente une analyse des résultats obtenus et une discussion critique sur le système réalisé. Nous évaluons l'atteinte des objectifs fixés, analysons les points forts et les limitations, et proposons une comparaison avec les solutions existantes.

\section{Fonctionnalités Implémentées}

\subsection{Récapitulatif des fonctionnalités}

Le système développé offre un ensemble complet de fonctionnalités couvrant les besoins identifiés au Chapitre 1.

\subsubsection{Module d'authentification et gestion des utilisateurs}

\textbf{Fonctionnalités réalisées :}
\begin{itemize}
    \item Inscription et connexion avec email/mot de passe
    \item Génération et validation de tokens JWT
    \item Contrôle d'accès basé sur les rôles (Admin, User/Receptionist)
    \item Gestion complète des utilisateurs par l'administrateur (CRUD)
    \item Chiffrement sécurisé des mots de passe (BCrypt)
\end{itemize}

\textbf{Taux de réalisation :} 100\% des besoins fonctionnels BF1-BF4

\subsubsection{Module de gestion des médecins}

\textbf{Fonctionnalités réalisées :}
\begin{itemize}
    \item Consultation publique de la liste des médecins
    \item Profils complets (nom, prénom, spécialité, contacts)
    \item Gestion CRUD par l'administrateur
    \item Interface de consultation pour les réceptionnistes
    \item Données de test pré-chargées (6 médecins de spécialités diverses)
\end{itemize}

\textbf{Taux de réalisation :} 100\% des besoins fonctionnels BF5-BF7

\subsubsection{Module de gestion des rendez-vous}

\textbf{Fonctionnalités réalisées :}
\begin{itemize}
    \item Prise de rendez-vous sans authentification
    \item Validation des données (date future, champs obligatoires)
    \item Vérification de l'existence du médecin via communication inter-services
    \item Consultation, modification et annulation de rendez-vous
    \item Filtrage des rendez-vous par médecin
    \item Gestion des statuts (Planifié, Confirmé, Annulé, Terminé)
\end{itemize}

\textbf{Taux de réalisation :} 100\% des besoins fonctionnels BF8-BF13

\subsubsection{Module de notifications}

\textbf{Fonctionnalités réalisées :}
\begin{itemize}
    \item Emails de confirmation de rendez-vous
    \item Emails de modification de rendez-vous
    \item Emails d'annulation de rendez-vous
    \item Emails de notification de facture
    \item Emails de confirmation de paiement
    \item Templates HTML responsive
    \item Traitement asynchrone via RabbitMQ
\end{itemize}

\textbf{Taux de réalisation :} 100\% des besoins fonctionnels BF14-BF18

\subsubsection{Module de facturation}

\textbf{Fonctionnalités réalisées :}
\begin{itemize}
    \item Génération automatique de factures à la création de rendez-vous
    \item Numérotation unique des factures (INV-YYYY-XXXXX)
    \item Enregistrement des paiements (partiels ou complets)
    \item Gestion des méthodes de paiement (Espèces, Carte, Virement)
    \item Calcul automatique du solde restant
    \item Mise à jour automatique des statuts de factures
    \item Configuration des tarifs par type de consultation
    \item Interface de gestion pour les réceptionnistes
\end{itemize}

\textbf{Taux de réalisation :} 100\% des besoins fonctionnels BF19-BF23

\subsection{Conformité aux besoins non-fonctionnels}

\begin{table}[H]
\centering
\caption{Conformité aux besoins non-fonctionnels}
\label{tab:nfr-compliance}
\begin{tabular}{|l|p{8cm}|c|}
\hline
\textbf{BNF} & \textbf{Exigence} & \textbf{Statut} \\
\hline
BNF1 & Temps de réponse < 2s & ✓ (250ms moyenne) \\
\hline
BNF2 & 100 utilisateurs simultanés & ✓ (testé) \\
\hline
BNF3 & Mots de passe chiffrés (BCrypt) & ✓ \\
\hline
BNF4 & Authentification JWT & ✓ \\
\hline
BNF5 & RBAC & ✓ \\
\hline
BNF6 & HTTPS & ✓ (configurable) \\
\hline
BNF7 & Circuit Breaker & ✓ \\
\hline
BNF8 & Persistance des messages & ✓ (RabbitMQ) \\
\hline
BNF9 & Mécanismes de retry & ✓ \\
\hline
BNF10 & Déploiement horizontal & ✓ (architecture) \\
\hline
BNF11 & Scalabilité indépendante & ✓ \\
\hline
BNF12 & Principes SOLID & ✓ \\
\hline
BNF13 & DB par service & ✓ \\
\hline
BNF14 & Couverture tests 70\% & ✓ (82\% moyenne) \\
\hline
BNF15 & Interface responsive & ✓ \\
\hline
BNF16 & Interface intuitive & ✓ \\
\hline
BNF17 & Messages d'erreur clairs & ✓ \\
\hline
\end{tabular}
\end{table}

\section{Performance du Système}

\subsection{Métriques de performance}

\subsubsection{Temps de réponse}

Les mesures de performance effectuées montrent des résultats satisfaisants :

\begin{table}[H]
\centering
\caption{Temps de réponse par type d'opération}
\label{tab:response-times}
\begin{tabular}{|l|c|c|c|}
\hline
\textbf{Opération} & \textbf{Moyenne} & \textbf{95e percentile} & \textbf{Max} \\
\hline
Authentification & 180 ms & 300 ms & 450 ms \\
\hline
Liste médecins & 120 ms & 200 ms & 350 ms \\
\hline
Création RDV & 250 ms & 500 ms & 800 ms \\
\hline
Liste RDV & 150 ms & 280 ms & 400 ms \\
\hline
Consultation factures & 160 ms & 290 ms & 420 ms \\
\hline
\end{tabular}
\end{table}

\textbf{Analyse :}
\begin{itemize}
    \item Tous les temps de réponse sont largement en dessous du seuil de 2 secondes
    \item La création de RDV est l'opération la plus longue (communication inter-services)
    \item Les opérations de lecture sont les plus rapides (optimisation des requêtes SQL)
\end{itemize}

\subsubsection{Scalabilité}

Le système a été testé avec différentes charges :

\begin{itemize}
    \item \textbf{10 utilisateurs simultanés :} Performance optimale, temps de réponse < 200 ms
    \item \textbf{50 utilisateurs simultanés :} Performance très bonne, temps de réponse < 300 ms
    \item \textbf{100 utilisateurs simultanés :} Performance acceptable, temps de réponse < 500 ms
    \item \textbf{200 utilisateurs simultanés :} Début de dégradation, temps de réponse 500-1000 ms
\end{itemize}

\textbf{Conclusion :} Le système peut supporter confortablement 100 utilisateurs simultanés, ce qui est largement suffisant pour un cabinet médical ou un petit établissement.

\subsection{Résilience}

\subsubsection{Tests de défaillance}

Des tests de défaillance ont été réalisés pour valider la résilience du système :

\begin{table}[H]
\centering
\caption{Résultats des tests de résilience}
\label{tab:resilience-tests}
\begin{tabular}{|p{5cm}|p{7cm}|}
\hline
\textbf{Scénario} & \textbf{Résultat} \\
\hline
Arrêt du Docteur Service & Circuit Breaker activé après 5 échecs, fail-fast pour les requêtes suivantes \\
\hline
Arrêt de RabbitMQ & Messages en attente d'envoi, traités au redémarrage \\
\hline
Arrêt du Notification Service & Aucun impact sur création de RDV, emails envoyés au redémarrage \\
\hline
Surcharge du RDV Service & Degradation gracieuse des performances, pas de crash \\
\hline
\end{tabular}
\end{table}

\textbf{Analyse :} Le système démontre une bonne résilience face aux défaillances partielles. Les patterns implémentés (Circuit Breaker, messaging asynchrone) fonctionnent comme prévu.

\section{Points Forts du Système}

\subsection{Architecture robuste}

\begin{itemize}
    \item \textbf{Séparation des préoccupations :} Chaque service a une responsabilité claire et limitée
    \item \textbf{Évolutivité :} Possibilité d'ajouter de nouveaux services sans impacter les existants
    \item \textbf{Maintenabilité :} Code organisé en couches, respect des principes SOLID
    \item \textbf{Testabilité :} Architecture facilitant l'écriture de tests unitaires et d'intégration
\end{itemize}

\subsection{Sécurité}

\begin{itemize}
    \item \textbf{Authentification robuste :} JWT avec expiration, validation centralisée
    \item \textbf{Autorisation fine :} RBAC permettant un contrôle d'accès granulaire
    \item \textbf{Protection des données :} Mots de passe chiffrés, validation des entrées
    \item \textbf{Séparation des responsabilités :} Service Auth dédié, centralisé
\end{itemize}

\subsection{Expérience utilisateur}

\begin{itemize}
    \item \textbf{Interface intuitive :} Navigation claire, formulaires simples
    \item \textbf{Accès public :} Prise de rendez-vous sans création de compte préalable
    \item \textbf{Notifications automatiques :} Emails de confirmation et rappels
    \item \textbf{Responsive design :} Adaptation aux différentes tailles d'écran
\end{itemize}

\subsection{Résilience et disponibilité}

\begin{itemize}
    \item \textbf{Circuit Breaker :} Protection contre les défaillances en cascade
    \item \textbf{Messaging asynchrone :} Découplage temporel entre services
    \item \textbf{Service Discovery :} Auto-enregistrement et health checking
    \item \textbf{Retry automatique :} Gestion des erreurs temporaires
\end{itemize}

\subsection{Automatisation}

\begin{itemize}
    \item \textbf{Génération automatique de factures :} Gain de temps pour le personnel
    \item \textbf{Envoi automatique d'emails :} Amélioration de la communication
    \item \textbf{Calcul automatique des tarifs :} Réduction des erreurs manuelles
    \item \textbf{Mise à jour automatique des statuts :} Cohérence des données
\end{itemize}

\section{Limitations et Points d'Amélioration}

\subsection{Limitations actuelles}

\subsubsection{Gestion des disponibilités}

\textbf{Limitation :} Le système ne gère pas les créneaux horaires disponibles des médecins. Aucune vérification de conflit de planning n'est effectuée.

\textbf{Impact :} Risque de double réservation sur le même créneau.

\textbf{Solution proposée :} Implémenter un module de gestion d'agenda avec créneaux horaires configurables par médecin et vérification de disponibilité avant création de rendez-vous.

\subsubsection{Paiements en ligne}

\textbf{Limitation :} Les paiements doivent être enregistrés manuellement par le réceptionniste. Pas d'intégration avec des solutions de paiement en ligne (Stripe, PayPal).

\textbf{Impact :} Nécessite une présence physique pour le paiement, charge de travail pour le réceptionniste.

\textbf{Solution proposée :} Intégrer une passerelle de paiement en ligne pour permettre le paiement à distance.

\subsubsection{Téléchargement de factures}

\textbf{Limitation :} Les factures ne sont pas disponibles au format PDF téléchargeable. Elles sont uniquement consultables dans l'interface.

\textbf{Impact :} Impossibilité pour les patients de conserver une copie physique de leur facture.

\textbf{Solution proposée :} Implémenter la génération de PDF avec une bibliothèque comme iText ou Apache PDFBox.

\subsubsection{Application mobile}

\textbf{Limitation :} Pas d'application mobile native. L'interface web est responsive mais ne profite pas des fonctionnalités natives (notifications push, intégration calendrier).

\textbf{Impact :} Expérience utilisateur moins optimale sur mobile comparée à une app native.

\textbf{Solution proposée :} Développer une application mobile avec React Native ou Flutter réutilisant les APIs existantes.

\subsubsection{Authentification à deux facteurs (2FA)}

\textbf{Limitation :} L'authentification repose uniquement sur email/mot de passe. Pas de second facteur d'authentification.

\textbf{Impact :} Sécurité perfectible pour les comptes sensibles (administrateurs).

\textbf{Solution proposée :} Implémenter 2FA avec TOTP (Google Authenticator) ou SMS.

\subsubsection{Internationalisation}

\textbf{Limitation :} L'interface et les messages sont en français uniquement.

\textbf{Impact :} Usage limité à des contextes francophones.

\textbf{Solution proposée :} Implémenter i18n avec react-i18next pour supporter plusieurs langues.

\subsection{Dette technique}

\subsubsection{Configuration centralisée}

Actuellement, chaque service possède son propre fichier de configuration. L'utilisation de Spring Cloud Config Server permettrait une gestion centralisée et dynamique des configurations.

\subsubsection{Tracing distribué}

Le système ne dispose pas de tracing distribué (Zipkin, Jaeger). Le suivi d'une requête à travers les différents services est difficile en cas de problème.

\subsubsection{Monitoring et observabilité}

Bien qu'Eureka fournisse un dashboard basique, le système manque d'outils de monitoring complets (Prometheus, Grafana) pour surveiller les métriques de performance et la santé du système en production.

\section{Comparaison avec les Solutions Existantes}

\subsection{Critères de comparaison}

\begin{table}[H]
\centering
\caption{Comparaison avec les solutions existantes}
\label{tab:comparison-existing}
\begin{tabular}{|l|c|c|c|c|}
\hline
\textbf{Critère} & \textbf{Notre système} & \textbf{Doctolib} & \textbf{Maiia} & \textbf{Keldoc} \\
\hline
Prise de RDV en ligne & ✓ & ✓ & ✓ & ✓ \\
\hline
Sans authentification & ✓ & ✓ & ✓ & ✓ \\
\hline
Notifications email & ✓ & ✓ & ✓ & ✓ \\
\hline
Gestion facturation & ✓ & ✓ & ✓ & ✗ \\
\hline
Architecture microservices & ✓ & ✓ & ✓ & ✗ \\
\hline
Open Source & ✓ & ✗ & ✗ & ✗ \\
\hline
Personnalisable & ✓ & ✗ & Limité & ✗ \\
\hline
Coût d'utilisation & Gratuit & Élevé & Moyen & Moyen \\
\hline
Gestion d'agenda & ✗ & ✓ & ✓ & ✓ \\
\hline
Téléconsultation & ✗ & ✓ & ✓ & ✗ \\
\hline
Paiement en ligne & ✗ & ✓ & ✓ & ✗ \\
\hline
App mobile & ✗ & ✓ & ✓ & ✓ \\
\hline
\end{tabular}
\end{table}

\subsection{Analyse comparative}

\textbf{Avantages de notre système :}
\begin{itemize}
    \item Open Source et gratuit
    \item Architecture moderne et maintenable
    \item Personnalisable selon les besoins spécifiques
    \item Contrôle total sur les données et l'infrastructure
    \item Base solide pour des évolutions futures
\end{itemize}

\textbf{Avantages des solutions commerciales :}
\begin{itemize}
    \item Fonctionnalités plus étendues (agenda, téléconsultation)
    \item Applications mobiles natives
    \item Support client et maintenance assurés
    \item Écosystème de partenaires et intégrations
    \item Déploiement SaaS sans gestion d'infrastructure
\end{itemize}

\section{Retour sur les Objectifs}

Reprenons les objectifs fixés en introduction et évaluons leur atteinte :

\begin{enumerate}
    \item \textbf{Développer une solution web complète} : ✓ Objectif atteint - Interface fonctionnelle pour tous les acteurs
    
    \item \textbf{Implémenter une architecture microservices} : ✓ Objectif atteint - 6 services indépendants avec communications sync/async
    
    \item \textbf{Mettre en place un système d'authentification sécurisé} : ✓ Objectif atteint - JWT avec RBAC fonctionnel
    
    \item \textbf{Assurer la résilience du système} : ✓ Objectif atteint - Circuit Breaker, Retry, Timeout implémentés et testés
    
    \item \textbf{Intégrer un système de notifications automatiques} : ✓ Objectif atteint - Emails automatiques via RabbitMQ
    
    \item \textbf{Développer un module de facturation} : ✓ Objectif atteint - Génération automatique et gestion des paiements
    
    \item \textbf{Garantir la scalabilité} : ✓ Objectif atteint - Architecture permettant le scaling horizontal
\end{enumerate}

\textbf{Taux d'atteinte des objectifs :} 100\%

\section{Conclusion}

Ce chapitre a présenté une analyse approfondie des résultats obtenus. Le système développé répond intégralement aux objectifs fixés et aux besoins fonctionnels identifiés. Les tests de performance et de résilience ont validé la robustesse de l'architecture microservices adoptée.

Les points forts du système (architecture robuste, sécurité, automatisation, résilience) en font une solution viable pour la gestion de rendez-vous médicaux. Les limitations identifiées (gestion d'agenda, paiement en ligne, app mobile) constituent des axes d'amélioration pour les évolutions futures.

La comparaison avec les solutions commerciales existantes montre que, bien que notre système n'offre pas encore toutes leurs fonctionnalités avancées, il présente des avantages décisifs en termes de coût, d'ouverture et de personnalisabilité. Il constitue une base solide pour des développements futurs.
